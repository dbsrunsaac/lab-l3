\documentclass[conference]{IEEEtran}
\IEEEoverridecommandlockouts
% The preceding line is only needed to identify funding in the first footnote. If that is unneeded, please comment it out.
\usepackage{cite}
\usepackage{amsmath,amssymb,amsfonts}
\usepackage{algorithmic}
\usepackage{graphicx}
\usepackage{textcomp}
\usepackage{xcolor}
\usepackage{tabularx}
\usepackage{multirow}
\usepackage{graphics} % for pdf, bitmapped graphics files
\usepackage{subfig}
\usepackage{subcaption}
\usepackage{hyperref}
\usepackage{academicons}
\usepackage{xcolor}
\usepackage{listings}
\usepackage{tabularx} % Asegúrate de incluir este paquete

\usepackage{tikz}
\usetikzlibrary{shapes.geometric, arrows}

\usetikzlibrary{shapes.geometric, arrows}

\tikzstyle{startstop} = [rectangle, rounded corners, minimum width=3cm, minimum height=1cm,text centered, draw=black, fill=red!30]
\tikzstyle{process} = [rectangle, minimum width=3cm, minimum height=1cm, text centered, draw=black, fill=blue!30]
\tikzstyle{arrow} = [thick,->,>=stealth]


\def\BibTeX{{\rm B\kern-.05em{\sc i\kern-.025em b}\kern-.08em
		T\kern-.1667em\lower.7ex\hbox{E}\kern-.125emX}}

% Color Enlace
\definecolor{colorEnlace}{RGB}{0, 0, 0}
\hypersetup{
	colorlinks=true,
	linkcolor=colorEnlace,
	citecolor=colorEnlace,
	urlcolor=colorEnlace,
	pdfauthor={Davis Bremdow Salazar Roa},
	pdftitle={Sistemas Embebidos}
}
\lstset{
	language=C,
	basicstyle=\ttfamily\small,
	keywordstyle=\color{blue},
	stringstyle=\color{red},
	commentstyle=\color{green!60!black},
	showstringspaces=false,
	numbers=left,
	numberstyle=\tiny\color{gray},
	frame=none,
	breaklines=true,
	tabsize=1
}

% Control 
\usepackage{amsmath}
\begin{document}
	
	\title{Informe final - Amplificador Diferencial Retroalimentado}
	\author{
		\makebox[\textwidth][c]{\large{Universidad Nacional de San Antonio Abad del Cusco}}\\
		\makebox[\textwidth][c]{\large\textit{Facultad de Ingeniería Electrónica}}\\[1ex]
		
		\IEEEauthorblockN{Ruth Juana Espino Puma - 184657}
		\IEEEauthorblockA{Estudiante de Ingeniería Electrónica \\
			Cusco, Perú \\
			200353@unsaac.edu.pe}
		\and
		\IEEEauthorblockN{Davis Bremdow Salazar Roa - 200353}
		\IEEEauthorblockA{Estudiante de Ingeniería Electrónica \\
			Cusco, Perú \\
			200353@unsaac.edu.pe}
	}
	\maketitle
	
	
	\begin{abstract}
		Los amplificadores retroalimentados son dispositivos electrónicos que incorporan una parte de la señal de salida de nuevo a su entrada con el objetivo de controlar y mejorar su comportamiento. Esta retroalimentación puede ser negativa, lo que estabiliza la ganancia, reduce la distorsión y amplía el ancho de banda, o positiva, que se utiliza en aplicaciones como osciladores. La retroalimentación negativa es ampliamente usada debido a sus beneficios en linealidad y estabilidad.
		
		Estos amplificadores son esenciales en diversas aplicaciones como amplificadores operacionales, filtros activos, sistemas de control automático, telecomunicaciones e instrumentación médica. Su importancia radica en que permiten un control preciso de parámetros eléctricos, mejorando el desempeño general del sistema y facilitando el diseño de circuitos confiables y predecibles.
	\end{abstract}
	
	\begin{IEEEkeywords}
		feedback, amplifier, gain, stability, bandwidth, distortion, negative, positive, op-amp, linearity
	\end{IEEEkeywords}
	
	% Contenido del documento
	
	
	\bibliographystyle{IEEEtran}
	\bibliography{biblio}
\end{document}